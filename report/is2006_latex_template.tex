\documentclass{article}
\usepackage{interspeech2006,amssymb,amsmath,epsfig}
\usepackage{hyperref} % For hyperlinks in the PDF

\usepackage{natbib} % Harvard style bib
\bibliographystyle{IEEEtranN}

\usepackage{lipsum} % Dummy text

\setcounter{page}{1}
\sloppy		% better line breaks
\ninept
%SM below a registered trademark definition
\def\reg{{\rm\ooalign{\hfil
     \raise.07ex\hbox{\scriptsize R}\hfil\crcr\mathhexbox20D}}}


\title{Dockerised Clouds: \\ A Comparative Study on Container Cluster Management Frameworks}

\makeatletter
\def\name#1{\gdef\@name{#1\\}}
\makeatother
\name{{\em Anthony Troy, Martin Somers and Marcelo Grossi}}

\address{School of Computing  \\
Dublin City University \\
Dublin 9, Ireland\\
{\small \tt\{anthony.troy3,martin.somers,marcelo.grossi2\}@mail.dcu.ie}
}


\begin{document}

\maketitle

\begin{abstract}
Containerisation is a recently resurged computing paradigm 
that is having a significant impact on how applications are being built, 
shipped and ran.\ Along with being less resource intensive and 
more portable, containers simplify dependency 
management, application versioning and service scaling, as opposed 
to deploying applications or application components directly 
onto a host operating system. Docker, albeit a relatively young project, 
has successfully established a container standard  
in Linux and poses itself as being production-ready. 
\par
Being interoperable in nature, container instances can be easily scaled
on a single host or across a cluster of hosts.
Increasingly, cluster management frameworks are providing 
first-class support for the Docker container standard and runtime. 
This includes established solutions such as Apache Mesos 
and Kubernetes.\ Docker now also includes its own native clustering
 tool, Swarm.
This paper reviews the current approaches and patterns forwarded by 
cluster management solutions to orchestrate and choreograph 
distributed container clusters under Docker. Subsequently, this paper
contributes a comparative analysis of two varying 
frameworks in this space,  Kubernetes and Swarm.


\end{abstract}

\section{Introduction}
Example citation \citep{db}. \lipsum[1]


\section{Related work}
\lipsum[1]

\section{Study Design}
\lipsum[1]
\begin{itemize}
\item \lipsum[1]
\end{itemize}

\subsubsection{Todo - Some title}
\lipsum[1]

\subsubsection{Todo - Another title}
\lipsum[1]


\section{Discussion}
\lipsum[1]

\section{Conclusions}
\lipsum[1]


\vspace{-7.5mm}
\renewcommand{\refname}{\section{References}}
\bibliography{is2006_latex_template}

\end{document}
