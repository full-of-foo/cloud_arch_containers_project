\documentclass{article}
\usepackage{interspeech2006,amssymb,amsmath,epsfig}
\usepackage{hyperref} % For hyperlinks in the PDF

\usepackage{natbib} % Harvard style bib
\bibliographystyle{IEEEtranN}

\usepackage{lipsum} % Dummy text

\setcounter{page}{1}
\sloppy		% better line breaks
\ninept
%SM below a registered trademark definition
\def\reg{{\rm\ooalign{\hfil
     \raise.07ex\hbox{\scriptsize R}\hfil\crcr\mathhexbox20D}}}


\title{Dockerised Clouds: \\ A Comparative Study on Container Cluster Management Frameworks}

\makeatletter
\def\name#1{\gdef\@name{#1\\}}
\makeatother
\name{{\em Anthony Troy, Martin Somers and Marcelo Grossi}}

\address{School of Computing  \\
Dublin City University \\
Dublin 9, Ireland\\
{\small \tt\{anthony.troy3,martin.somers,marcelo.grossi2\}@mail.dcu.ie}
}


\begin{document}

\maketitle

\begin{abstract}
Containerisation is a recently resurged computing paradigm 
that is having a significant impact on how applications are being built, 
shipped and ran.\ Being interoperable in nature, container instances can be easily scaled
on a single host or across a cluster of hosts.\
Docker, albeit a relatively young project, 
has successfully established a container standard  
in Linux and poses itself as being production-ready.\ 
Increasingly, cluster management frameworks are providing 
first-class support for the Docker container standard and runtime. 
This includes established solutions such as Apache Mesos 
and Kubernetes.\ Docker now also includes its own native clustering
 tool, Swarm.
This paper considers Docker's production-readiness with respect
to native clustering capabilities and runtime interoperability.\
We review the current approaches and patterns forwarded by 
cluster management solutions to orchestrate distributed container clusters under Docker,\
and subsequently contribute a comparative analysis of two varying 
frameworks in this space,  Kubernetes and Swarm.


\end{abstract}

\section{Introduction}
Containers have a long history in computing though much of their recent popularity 
surround the recent developments of both LXC and the Docker platform. 
The former can be described as a container execution environment,
or more formally, a Linux user space interface to 
access new kernel capabilities of achieving process isolation through namespaces
and cgroups \citep{Claus}. The latter is an open-source suite of tools managed by Docker Inc.\ which
extends upon container technology such as LXC, in turn 
allowing containers to behave like ``full-blown hosts in their own right" 
whereby containers have ``strong isolation, their own network and storage stacks, as well 
as resource management capabilities to allow friendly co-existence of multiple containers on a host" \citep{db}.
\par 
Uncertainties around Docker's maturity and production-readiness have been expressed \citep{Kereki, Powers, Merkel}, however 
over the last two years the states of both Docker and the containerisation ecosystem continue to rapidly progress.\
Last year Docker has seen an unprecedented increase in development, adoption and community uptake \citep{Merkel}. Most
notably was the introduction of customisable container execution environments. This means as opposed to LXC one can
``take advantage of the numerous isolation tools available" such as ``OpenVZ, systemd-nspawn, libvirt-sandbox, qemu/kvm, BSD Jails and Solaris Zones".
Also included in this 0.9 release was the new built-in container execution driver ``libcontainer", which replaced LXC as the default driver.
Going forward on all platforms Docker can now execute kernel features such as ``namespaces, control groups, capabilities, apparmor profiles, 
network interfaces and firewalling rules" predictably ``without depending on LXC" as an external dependency \citep{Hykes}. 
\par
Interestingly, libcontainer itself was the first project to provide a standard interface for making containers and managing their lifecycle.\
Subsequently the Docker CEO  announced the coming together of industry leaders and others in partnership with the Linux Foundation
to form a ``minimalist, non-profit, openly governed project" named The Open Container Initiative (OCI), with the purpose of defining 
``common specifications around container format and runtime" \citep{Golub}. 
Thereafter Docker donated its base container format and runtime, libcontainer, to be maintained by the OCI. 
\par
Amidst establishing a container standard, Docker has made significant headway in 
supporting multi-host cloud production environments. In terms of native tooling, in the last year Docker has implemented
a suite of tools for provisioning and orchestrating containers:
\begin{itemize}
\item \textbf{Docker Machine} allows one to provision Docker hosts, which are simply Linux virtual machines (VMs) supporting Docker, on a local machine or cloud. 
Its pluggable driver API currently supports ``provisioning Docker locally with Virtualbox as well as remotely" on cloud providers such Digital Ocean, AWS, Azure and VMware.
\item \textbf{Docker Swarm} is a clustering solution which takes the standard 
``Docker Engine and extends it to enable you to work on a cluster of containers". 
This in turn allows one to ``manage a resource pool of Docker hosts and schedule
containers to run transparently on top, automatically managing workload and providing failover services".
\item \textbf{Docker Compose} is the ``glue" allowing one to compose a multi-host application on top of a Swarm cluster whereby you
can specify how each application is to be ran in the the cluster, in turn allowing one to orchestrate and choreograph local or cloud containers.
\end{itemize}
\noindent In many cases an existing cloud infrastructure depends upon one or more orchestration tools, for example 
Consul for service discovery. Typically, such tools cannot be migrated away from easily and in turn cause ``vendor lock-in".
Consequently, Docker have implemented this trio of orchestration tools in a generic way, 
providing ``a standard interface to service providers so that they can almost be used as plug-and-play solutions" on top of the Docker platform \citep{holla}.
\par
Backed by both the industry and community, Docker is now presented as a more mature and production-ready
platform, however it should be mentioned that there yet exists no
formal topology for a fully, or partly, Dockerised cloud \citep{Claus}. In 
turn, there is no widely accepted solution for managing distributed 
Docker-based clusters.\ This is unsurprising given that 
practitioners and industry experts have noted that frameworks in this space
vary greatly in terms of capability, architecture and target cluster proportion
\citep{goasguen, holla}.
\par
This paper presents TODO. The remainder of this paper is structured as follows. TODO


\section{Study Design}
\lipsum[1]
\begin{itemize}
\item \lipsum[1]
\end{itemize}

\subsubsection{Todo - Some title}
\lipsum[1]

\subsubsection{Todo - Another title}
\lipsum[1]


\section{Discussion}
\lipsum[1]

\section{Conclusions}
\lipsum[1]


\vspace{-7.5mm}
\renewcommand{\refname}{\section{References}}
\bibliography{is2006_latex_template}

\end{document}
